\documentclass{article}
\usepackage[utf8]{inputenc}
\usepackage[english]{babel}
\usepackage{amsmath}
\usepackage{amssymb}
\usepackage{graphicx}
\usepackage{listings}
\usepackage{color}
\usepackage{amsthm}
\usepackage{lipsum}
\usepackage{enumitem} 

\newtheorem{theorem}{Theorem}
\newenvironment{amatrix}[1]{%
  \left(\begin{array}{@{}*{#1}{c}|c@{}}
}{%
  \end{array}\right)
}

\newlist{answer}{enumerate}{1}
\setlist[answer]{
  label = \textbf{Answer \arabic*.}, 
  wide
}
\newlist{negated}{enumerate}{1}
\setlist[negated]{
  label = \textbf{Negated Theorem \arabic*.}, 
  wide
}

\renewcommand{\qedsymbol}{Q.E.D.}
\newcommand{\norm}[1]{\left\lVert#1\right\rVert}
\newcommand{\inpro}[2]{\langle#1,#2\rangle}
\newcommand{\proj}[2]{\text{proj}_{#2}\left(#1\right)}

\definecolor{dkgreen}{rgb}{0,0.6,0}
\definecolor{gray}{rgb}{0.5,0.5,0.5}
\definecolor{mauve}{rgb}{0.58,0,0.82}
\lstset{frame=tb,
  language=Java,
  aboveskip=3mm,
  belowskip=3mm,
  showstringspaces=false,
  columns=flexible,
  basicstyle={\small\ttfamily},
  numbers=none,
  numberstyle=\tiny\color{gray},
  keywordstyle=\color{blue},
  commentstyle=\color{dkgreen},
  stringstyle=\color{mauve},
  breaklines=true,
  breakatwhitespace=true,
  tabsize=3
}

\title{CS 241 Homework 2}
\author{Christian Gutierrez}
\date{Spring 2022}

\begin{document}

\maketitle

\newpage
%problem 1
\begin{theorem}
    $\forall e_{1},e_{2},e_{1}*e_{2}$ is even
\end{theorem}
\begin{negated}
  \item $\exists e_{1},e_{2},e_{1}*e_{2}$ is odd
\end{negated}
\begin{proof}
    Prove P:\\
    $e_{1} = 2n$ for some integer n \\
    $e_{1} = 2m$ for some integer m \\
    $e_{1}*e_{2}=2n*2m=2(2nm)=2k$ for some integer k\\
    Clearly the product is even due to being divisible by 2\\
\end{proof}

%problem 2
\begin{theorem}
  Prove $\bar{P}$:\\
  $\exists d, e| e^{d}$ is odd
\end{theorem}
\begin{negated}
\item $\forall d, e| e^{d}$ is even
\end{negated}
\begin{proof}
  $e^{d} = e*e*e\ldots e$ d times \\
  which is a product containing 2 as a factor, so it is even by definition.
\end{proof}

%problem 3
\begin{theorem}
  $\forall a,b\exists c | a^{2}*b^{2}=c^{2}$
\end{theorem}
\begin{negated}
\item $\exists a,b \forall c | a^{2}*b^{2}=c^{2}$
\end{negated}
\begin{proof}
  Prove P:\\
  $a^{2}*b^{2}=(a*b)^{2} = c^{2}$\\
  $c=a*b$ so c exists
\end{proof}

%problem 4
\begin{theorem}
  $\exists c \forall a,b  | a^{2}*b^{2}=c^{2}$
\end{theorem}
\begin{negated}
\item $\forall c \exists a,b  | a^{2}*b^{2}=c^{2}$
\end{negated}
\begin{proof}
  Prove $\bar{P}$:\\
  $a^{2}*b^{2}=(a*b)^{2} = c^{2}$\\
  $ab=c$ so a and b exist
\end{proof}


% Copy these segments for each problem
%Suppose that there were a finite number of primes such that $|P|=n$. Consider the product of all prime numbers
%    \begin{equation}
%        k = p_1p_2p_3\ldots p_n.
%    \end{equation}
    % Clearly $k$ is divisible by all prime numbers. Now consider the number $k+1$. Since $k$ is divisible by all primes, $k+1$ cannot be divisible by any primes, which means that $k+1$ is prime. However $k+1$ is not in our set $P$, which contradicts our claim that $P$ contained all primes. $\therefore$ there must be infinite number of primes. 
\newpage


\end{document}