\documentclass{article}
\usepackage[utf8]{inputenc}
\usepackage[english]{babel}
\usepackage{amsmath}
\usepackage{amssymb}
\usepackage{graphicx}
\usepackage{listings}
\usepackage{color}
\usepackage{amsthm}
\usepackage{lipsum}
\usepackage{enumitem}

\newtheorem{theorem}{Problem}
\newenvironment{amatrix}[1]{%
  \left(\begin{array}{@{}*{#1}{c}|c@{}}
}{%
  \end{array}\right)
}

\newlist{answer}{enumerate}{1}
\setlist[answer]{
  label = \textbf{Answer \arabic*.}, 
  wide
}

\renewcommand{\qedsymbol}{Q.E.D.}
\newcommand{\norm}[1]{\left\lVert#1\right\rVert}
\newcommand{\inpro}[2]{\langle#1,#2\rangle}
\newcommand{\proj}[2]{\text{proj}_{#2}\left(#1\right)}

\definecolor{dkgreen}{rgb}{0,0.6,0}
\definecolor{gray}{rgb}{0.5,0.5,0.5}
\definecolor{mauve}{rgb}{0.58,0,0.82}
\lstset{frame=tb,
  language=Java,
  aboveskip=3mm,
  belowskip=3mm,
  showstringspaces=false,
  columns=flexible,
  basicstyle={\small\ttfamily},
  numbers=none,
  numberstyle=\tiny\color{gray},
  keywordstyle=\color{blue},
  commentstyle=\color{dkgreen},
  stringstyle=\color{mauve},
  breaklines=true,
  breakatwhitespace=true,
  tabsize=3
}

\title{CS 241 Homework 1}
\author{Christian Gutierrez}
\date{Spring 2022}

\begin{document}

\maketitle

\newpage
\begin{theorem}
    On January 22nd 2020, I turned 22 years old. Prove that for
any person, there is exactly one year in which they turn x years old on the x
day of a month.
\end{theorem}

\begin{answer}
    \item Let
  \begin{equation}
    x\epsilon X =[1,31]\cap Z={[1,2,3,\ldots,31]}
  \end{equation}
  be the day of the month on which I was born. Let 
    Let P(x)="I will turn x years old on day x" Suppose P(x) and P(y)
    are both true with x!=y. Then I have two diffrent birthdays, the x'th and the y'th, which is
    impossible. Therefore if P(x) is true then it is true
    for a unique x. My age in years takes on all possible values in X,
    therfore there exists x such that P(x) is true.

\end{answer}

\begin{theorem}
    Clearly the following proof must be incorrect, where and what is
the error?
\end{theorem}

\begin{answer}
    \item
    \begin{equation}
        \frac{d}{dx}[x+x+x+ \ldots +x] x times
    \end{equation}

    \begin{equation}
    f'(x)=\lim_{h\to0}\frac{f(x+h)-f(x)}{h}
  \end{equation}
  \begin{equation}
    x+x+x+\ldots+x =\sum_{i=1}^{x} x 
  \end{equation}
  \begin{equation}
    \lim_{h\to0}=\frac{\sum_{i=1}^{x+h} (x+h)-\sum_{i=1}^{x}(x)}{h}
  \end{equation}
  \begin{equation}
    \lim_{h\to0}=\frac{\sum_{i=1}^{x+h}(x)+\sum_{i=1}^{x+h}(h)+\sum_{i=1}^{x}(x)}{h}
  \end{equation}
  \begin{equation}
    \lim_{h\to0}=\frac{\sum_{i=1}^{x}(x) + \sum_{i=x+1}^{x+h}(x)+\sum_{i=1}^{x+h}(h)-\sum_{i=1}^{x}(x)}{h}
  \end{equation}
  \begin{equation}
    \lim_{h\to0}=\frac{\sum_{i=x+1}^{x+h}(x)  + \sum_{i=1}^{x+h}(h)}{h}
  \end{equation}
  \begin{equation}
    \lim_{h\to0}=\frac{(x+x+x+\ldots+x+x)+(h+h+h+\ldots+h+h+h)}{h} x+h-x times, x+h times
  \end{equation}
  \begin{equation}
    \lim_{h\to0}=\frac{(x+x+x+\ldots+x+x)+(h+h+h+\ldots+h+h+h)}{h} h times, x+h times
  \end{equation}
  \begin{equation}
    \lim_{h\to0}=\frac{h*x+(x+h)*h}{h}
  \end{equation}
  \begin{equation}
    \lim_{h\to0}=\frac{h(x+x+h)}{h}
  \end{equation}
  \begin{equation}
    \lim_{h\to0}=\frac{h(x+x+h)}{h}
  \end{equation}
  \begin{equation}
    \lim_{h\to0}=(x+x+h)
  \end{equation}
  \begin{equation}
    =x+x+0
  \end{equation}
  \begin{equation}
    2x=2x
  \end{equation}
\end{answer}

\begin{proof}
    Clearly $1=2$ is is false due it containing diffrent numbers
    and inbetween steps 2, and 3 forgetting the limit definiton of a derivitve.
    \[ \lim_{h\to0} \frac{f(x+h)-f(x)}{h} \] 
    \end{proof}
% Copy these segments for each problem

% Suppose that there were a finite number of primes such that $|P|=n$. Consider the product of all prime numbers
%     \begin{equation}
%         k = p_1p_2p_3\ldots p_n.
%     \end{equation}
\newpage


\end{document}