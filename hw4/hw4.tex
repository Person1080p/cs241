\documentclass{article}
\usepackage[utf8]{inputenc}
\usepackage[english]{babel}
\usepackage{amsmath}
\usepackage{amssymb}
\usepackage{graphicx}
\usepackage{listings}
\usepackage{color}
\usepackage{amsthm} 

\newtheorem{theorem}{Theorem}
\newenvironment{amatrix}[1]{%
  \left(\begin{array}{@{}*{#1}{c}|c@{}}
}{%
  \end{array}\right)
}
\renewcommand{\qedsymbol}{Q.E.D.}
\newcommand{\norm}[1]{\left\lVert#1\right\rVert}
\newcommand{\inpro}[2]{\langle#1,#2\rangle}
\newcommand{\proj}[2]{\text{proj}_{#2}\left(#1\right)}

\definecolor{dkgreen}{rgb}{0,0.6,0}
\definecolor{gray}{rgb}{0.5,0.5,0.5}
\definecolor{mauve}{rgb}{0.58,0,0.82}
\lstset{frame=tb,
  language=Java,
  aboveskip=3mm,
  belowskip=3mm,
  showstringspaces=false,
  columns=flexible,
  basicstyle={\small\ttfamily},
  numbers=none,
  numberstyle=\tiny\color{gray},
  keywordstyle=\color{blue},
  commentstyle=\color{dkgreen},
  stringstyle=\color{mauve},
  breaklines=true,
  breakatwhitespace=true,
  tabsize=3
}

\title{CS 241 Homework 4}
\author{Christian Gutierrez}
\date{Spring 2022}

\begin{document}

\maketitle

\newpage

\begin{theorem}
    Let $n$ be an integer such that $ n \geq 2$. Either $n$ is prime, or n can be expressed as a unique product of primes.
\end{theorem}
\begin{proof}
    Let $P(n)$ where $n$ is either prime or can be expressed as a unique product of primes \\
    $P(n)$ holds all integers where $n \geq 2$\\\\
    Base: $P(2)$ is true becuase 2 is a number.\\\\
    Induction: Let $P(k)$ be true for all integers $k$ with $2 \leq k \leq n$\\
    Consider $n+1$. It is either prime or divisible by some number between 2 and n.\\
    If $n+1$ is prime, then $P(n+1)$ is true.\\
    If $n+1$ is divisible by some number, then suppose $n+1 = k*m$ where $k$ and $m$ are integers between $2$ and $n$.\\
    As a result, $P(k)$ and $P(m)$ are true by the Inductive Hypothesis.\\
    Then $k$ and $m$ are a product of primes or are prime.\\
    So $k=p_{1}*\ldots * p_{r} $ and $ m = q_{1}*\ldots q_{s}$ where $p_{i}$ and $q_{i}$ are prime, and $r,s \geq 1$.\\
    So $n+1= k * m = p_{1}*\ldots * p_{r} * q_{1}* \ldots * Q_{s}$ so $n+1$ is a product of primes.\\
    Therefore, $P(n)$ is true for all $n \geq 2$
  \end{proof}

% Copy these segments for each problem

\newpage

\begin{theorem}
  Let $ A \subseteq \mathbb{N}$ and $a \ne \phi$. $\exists a_{min} \in A 
  | \forall a \ne a_{min} \in A, a_{min} < a$ 
\end{theorem}
\begin{proof}
  Base: $n=1$. If $1 \in A$, then $1$ is the smallest element.\\\\
  Induction: Assume $P(1),\ldots, P(k),\ldots,P(n)$ and $P(n+1)$\\
  Suppose set $A \subset N $ where $A$ is non empty and contains $n+1$ as an element.\\
  So either some number from $1,\ldots, n \in A$ or not.\\
  If $k$ is in $A$ where $1 \leq k \leq n$ then $A$ is true by the Inductive Hypothesis\\
  As a result, $A$ has a minimum element.
  
\end{proof}



\end{document}