\documentclass{article}
\usepackage[utf8]{inputenc}
\usepackage[english]{babel}
\usepackage{amsmath}
\usepackage{amssymb}
\usepackage{graphicx}
\usepackage{listings}
\usepackage{color}
\usepackage{amsthm} 

\newtheorem{theorem}{Theorem}
\newenvironment{amatrix}[1]{%
  \left(\begin{array}{@{}*{#1}{c}|c@{}}
}{%
  \end{array}\right)
}
\renewcommand{\qedsymbol}{Q.E.D.}
\newcommand{\norm}[1]{\left\lVert#1\right\rVert}
\newcommand{\inpro}[2]{\langle#1,#2\rangle}
\newcommand{\proj}[2]{\text{proj}_{#2}\left(#1\right)}

\definecolor{dkgreen}{rgb}{0,0.6,0}
\definecolor{gray}{rgb}{0.5,0.5,0.5}
\definecolor{mauve}{rgb}{0.58,0,0.82}
\lstset{frame=tb,
  language=Java,
  aboveskip=3mm,
  belowskip=3mm,
  showstringspaces=false,
  columns=flexible,
  basicstyle={\small\ttfamily},
  numbers=none,
  numberstyle=\tiny\color{gray},
  keywordstyle=\color{blue},
  commentstyle=\color{dkgreen},
  stringstyle=\color{mauve},
  breaklines=true,
  breakatwhitespace=true,
  tabsize=3
}

\title{CS 241 Homework X}
\author{Christian Gutierrez}
\date{Spring 2022}

\begin{document}

\maketitle

\newpage

\begin{theorem}[There are Infinite Primes]
    Let $P$ be the set of all prime numbers. $\forall n\in\mathbb{N}$, $|P|>n$.
\end{theorem}
\begin{proof}
    Suppose that there were a finite number of primes such that $|P|=n$. Consider the product of all prime numbers
    \begin{equation}
        k = p_1p_2p_3\ldots p_n.
    \end{equation}
    Clearly $k$ is divisible by all prime numbers. Now consider the number $k+1$. Since $k$ is divisible by all primes, $k+1$ cannot be divisible by any primes, which means that $k+1$ is prime. However $k+1$ is not in our set $P$, which contradicts our claim that $P$ contained all primes. $\therefore$ there must be infinite number of primes. 
\end{proof}

% Copy these segments for each problem

\newpage


\end{document}