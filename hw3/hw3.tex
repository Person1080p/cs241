\documentclass{article}
\usepackage[utf8]{inputenc}
\usepackage[english]{babel}
\usepackage{amsmath}
\usepackage{amssymb}
\usepackage{graphicx}
\usepackage{listings}
\usepackage{color}
\usepackage{amsthm} 

\newtheorem{theorem}{Theorem}
\newenvironment{amatrix}[1]{%
  \left(\begin{array}{@{}*{#1}{c}|c@{}}
}{%
  \end{array}\right)
}
\renewcommand{\qedsymbol}{Q.E.D.}
\newcommand{\norm}[1]{\left\lVert#1\right\rVert}
\newcommand{\inpro}[2]{\langle#1,#2\rangle}
\newcommand{\proj}[2]{\text{proj}_{#2}\left(#1\right)}

\definecolor{dkgreen}{rgb}{0,0.6,0}
\definecolor{gray}{rgb}{0.5,0.5,0.5}
\definecolor{mauve}{rgb}{0.58,0,0.82}
\lstset{frame=tb,
  language=Java,
  aboveskip=3mm,
  belowskip=3mm,
  showstringspaces=false,
  columns=flexible,
  basicstyle={\small\ttfamily},
  numbers=none,
  numberstyle=\tiny\color{gray},
  keywordstyle=\color{blue},
  commentstyle=\color{dkgreen},
  stringstyle=\color{mauve},
  breaklines=true,
  breakatwhitespace=true,
  tabsize=3
}

\title{CS 241 Homework 3}
\author{Christian Gutierrez}
\date{Spring 2022}

\begin{document}

\maketitle

\newpage
%problem 1
\begin{theorem}
    Let $A$,$B$ be sets. $A \triangle B = \phi$ iff $A = B$
\end{theorem}
\begin{proof}
    The symmetric diffrence of $A$ and $B$ is \\
    $A \Delta B = (A \cup B)-(A \cap B)$ \\
    $A \Delta B = (A - B) \cup (B - A)$ \\  \\
    Suppose $A \Delta B$ = EMPTY \\
    Suppose $x$ is some variable and that x $\in$ A.\\
    I claim that $x \in B$. To suppose a contradiction that $x \not\in B$.\\
    So that $x \in A \cup B$ but $x \notin A \cap B$ \\
    So $x \in A \Delta B$, which results in $x \in$ EMPTY. \\
    This results in a contradiction, so $x \in B$ \\
    So, $A \subseteq B$.\\ \\
    For the opposite direction:\\
    Suppose $A \Delta B$ = EMPTY \\
    Suppose $x$ is some variable and that x $\in$ B.\\
    I clain that $x \in A$. To suppose a contradiction that $x \not\in A$.\\
    So that $x \in B \cup A$ but $x \notin B \cap A$ \\
    So $x \in B \Delta A$, which results in $x \in$ EMPTY. \\
    This results in a contradiction, so $x \in A$ \\
    So, $B \subseteq A$. \\

    Therefore: $A = B$.
\end{proof} \newpage

% problem 2
\begin{theorem}
  Let $S_n$= \{$x \in \mathbb{R} | 0 \leq x \leq \frac{n-1}{n}$\}. Then
  \begin{displaymath}
    \bigcup_{n=1}^{\infty}{S_n} = [0,1)
  \end{displaymath}
\end{theorem}
\begin{proof}
  $\bigcup_{n=1}^{\infty}{S_n} = \{x: \exists n \in N$ such that $x \in S_{n}\}$ \\
  Let $x\in \bigcup_{n=1}^{\infty}{S_n}$. Then for some $n, x \in \{ x \in \mathbb{R}: 0 \leq x \leq \frac{n-1}{n}\}$.\\
  Then $0 \leq x \leq \frac{n-1}{n}$. But$\frac{n-1}{n} < 1 $. \\
  So, $0 \leq x < 1$. And $x \in [0,1)$\\
  This shows LHS $\subseteq RHS$.\\ \\
  Now suppose $x \in [ 0,1 )$\\
  Then $0 \leq x< 1$\\
  I claim that there exists n, such that $x \leq \frac{n-1}{n}$.\\
  For $x \leq 1 - \in$ for some $\in >0$\\ \\
  Then choose $n > 1 \frac{1}{\in}$\\
  $\in > \frac{1}{n}$\\
  $\frac{1-1}{n}>1-\in$\\
  $\frac{n-1}{n}> 1-\in$\\ \\
  Then $\frac{n-1}{n}>1-\in \leq x$ for this n.\\
  So, $0\leq x > \frac{n-1}{n}$.\\
  Then $x \in \{ x \in\mathbb{R} | 0 \leq x \leq \frac{n-1}{n}\}$.\\ 
  So $x\in \bigcup_{n=1}^{\infty}{S_n} = [0,1)$
\end{proof}



% Copy these segments for each problem

\newpage


\end{document}